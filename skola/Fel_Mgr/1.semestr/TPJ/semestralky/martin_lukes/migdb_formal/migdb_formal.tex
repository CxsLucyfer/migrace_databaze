%%This is a very basic article template.
%%There is just one section and two subsections.
\documentclass{article}
\usepackage[utf8]{inputenc}              	%nastavení výchozího kódování textu
\usepackage[czech]{babel}                	%nastavení českých znaků
\usepackage[pdftex]{graphicx}				%umožňuje vkládání obrázků v jpg, png
\usepackage{amsfonts}						%package obsahujici symboly mnozin treba - \mathbb{N}
\begin{document}


\section{Úvod}
Tato práce byla vypracována v rámci předmětu TPJ jako formální specifikace
projektu Migdb a ověření správnosti jejího fragmentu.

\section{Neformální nastínění}
Programátoři v rámci dekompozice rozdělili aplikaci do několika layers - vrstev.
Dvě z těchto vrstev jsou aplikační a databázová vrstva. ORM mapování převádí
data a struktury nutné pro uložení těchto dat do databáze.

Projekt Migdb se staví k tomuto úkolu definováním dvou metamodelů - aplikačního
a databázového. Instance prvního metamodelu definuje
strukturu aplikace, které třídy je možné definovat, s jakými atributy atp. Instance druhého modelu
potom definuje strukturu databáze, tudíž tabulky a sloupce, do kterých je možné
ukládat data. Bez jakýchkoliv dalších features by tyto modely definovaly
klasický vztah ORM, nicméně Migdb si dalo za cíl definovat možné změny na úrovni
obou modelů. Tyto změny, které simulují genezi aplikace v průběhu software
development procesu jsou popsány Operacemi na úrovni obou modelů, které
spadají do zmiňovaných metamodelů. Vztah mezi těmito dvěmi Migdb metamodely
definuje Migdb ORM. V následujícím textu bude slovo ORM použito ve významu Migdb
ORM.

Poznámka: Ačkoliv projekt Migdb v nynější fázi zvažuje problém s ohledem na
existující instance dat v databázi, instance budou v rámci této práce zanedbány
a problém bude přehodnocen v budoucnu.


\section{Formální popis - gramatika aplikačního modelu}
Terminály jsou tučně \\

$App \rightarrow (AOp*\ |\ AResult )$ \\
$AOp \rightarrow AddClass\ |\ RemoveClass\ |\ AddProperty\ |\ RemoveProperty\ |\
SetParent $ \\
$AResult \rightarrow$ \textbf{Error} $|\ AGen )$ \\
$AddClass \rightarrow $ \textbf{addClass (} $String$\textbf{)} \\
$RemoveClass \rightarrow $ \textbf{removeClass (} $String$\textbf{)} \\
$AddProperty \rightarrow $ \textbf{addProperty (} $String, Type $\textbf{)} \\
$RemoveProperty  \rightarrow $ \textbf{removeProperty (} $String,
String$\textbf{)} \\
$Type \rightarrow $ \textbf{AInt | AString} \\
$SetParent \rightarrow setParent(String, String)$ \\
$String \rightarrow [a-s]*[A-S]* $ \\

\section{Formální popis - gramatika databázového modelu}
$Db \rightarrow (DOp* | DResult )$ \\
$DOp \rightarrow AddTable \ |\ RemoveTable\ |\ AddColumn\ |\ RemoveColumn $ \\
$DResult \rightarrow$ \textbf{DError} $|\ DGen )$ \\
$AddTable \rightarrow $ \textbf{addTable (} $DString$\textbf{)} \\
$RemoveTable \rightarrow $ \textbf{removeTable (} $DString$\textbf{)} \\
$AddColumn \rightarrow $ \textbf{addColumn (} $DString, DType $\textbf{)} \\
$RemoveColumn  \rightarrow $ \textbf{removeProperty (} $DString,
DString$\textbf{)} \\
$DType \rightarrow $ \textbf{DInt | Varchar} \\
$DString \rightarrow\ [a-s]* $

\section{Denotační sémantika}
   \textit{viz db.maude}
   
   
\end{document}


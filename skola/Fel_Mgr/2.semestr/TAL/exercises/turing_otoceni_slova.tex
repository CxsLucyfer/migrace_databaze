%%This is a very basic article template.
%%There is just one section and two subsections.
\documentclass{article}
\usepackage[czech, english]{babel}
\usepackage[T1]{fontenc} % pouzije EC fonty \usepackage[utf8]{inputenc}
\usepackage{graphicx} 
\usepackage[utf8]{inputenc}
\usepackage{amsmath}  %package potrebny pro matiku
\usepackage{amsfonts}  %package obsahujici symboly mnozin 
\begin{document}

\section{Vypracoval}
Martin Lukeš \\
aka woxie\\
21.6. 2012

\section{Zadání}
Navrhněte deterministický Turingův stroj s jednou páskou, který realizuje
funkci f(w)= $w^r$, kde $w^r$ je reverze slova w, $w \in \{0,1\}^*$. Nejprve
vysvětlete princip, pak popište jeho přechodovou funkci buď tabulkou nebo
stavovým diagramem.) Zdůvodněte, že navržený Turingův stroj funkci opravdu
realizuje.

\section{Princip}
\begin{enumerate}
  \item  Stroj dopíše na konec slova písmeno Z, které odlišuje původní slovo od
výstupu.
  \item  Následně bude kopírovat písmena z konce slova za Z, čímž vznikne
otočeného slova kopie slova. 
	\begin{enumerate}
		\item Jako zarážku bude v původním slově použivat
očárkovanou verzi písmene, v otočeném slově není nutná zarážka, písmena se budou
kopírovat na místo prvního blank symbolu (B) za slovem.
		\item v každé iteraci se provede zkopírování očárkovaného písmene do části za
		Z, následně se při návratu posune index kopírovaného písmene doleva
		\item kopírování skončí, jakmile se TM dostane při posunu očárkovaného písmene
		na blank symbol		
	\end{enumerate}
\item  Poté, co je celé původní slovo zkopírované dojde k smazání počátku (původního
 slova a symbolu Z).
\end{enumerate}

Daný stroj funkci realizuje - invariant - v každé iteraci jsou zkopírovaná
všechna písmena mezi očárkovaným písmenem a Z v reverzním pořadí. Variant -
konečnost - energie - algoritmus je konečný, jako variant zvolíme vzdálenost
mezi prvním písmenem slova a očárkovaným písmenem. Tato vzdálenost se v každé
iteraci (2) snižuje o jedna, až následně se dostane na hodnotu 0, kdy se
kopíruje první písmeno slova a algoritmus končí.

\section{Tabulka}

\begin{tabular}{|c|c|c|c|c|c|c|c|}
\hline
 stav & popis & 0 & 1 & B & $0'$ & $1'$ & Z \\
 \hline
 \hline
 $q_0$ & init/alespoň jeden znak & $q_1 0R$ & $q_1 1R$ & $q_f BR$ & & & \\
 \hline
 $q_1$ & přeskoč na konec slova & $q_1 0R$ & $q_1 1R$ & $q_2 ZL$ & & & \\
 \hline
 $q_2$ & if 0 - copy 0 if 1 copy 1 else break & $q_3 0'R$ & $q_5 1'R$ & $q_6 BR$
 & & & \\
 \hline
 $q_3$ & zkopíruj 0 & $q_3 0R$ & $q_3 1R$ & $q_4 0L$ & & & $q_3 ZR$\\
 \hline
 $q_4$ & návrat do ifu & $q_4 0L$ & $q_4 1L$ & & $q_2 0L$ & $q_2 1L$ & $q_4 ZL$
 \\
 \hline
 $q_5$ & zkopíruj 1 & $q_5 0R$ & $q_5 1R$ & $q_4 1L$ & & & $q_5 ZR$\\
 \hline
 $q_6$ & smaž začátek & $q_6 BR$ & $q_6 BR$ & & & & $q_f BR$ \\
 \hline
 $q_f$ & konec & & & & & & \\
 \hline 
\end{tabular}

\section{Výpočet na instancích}
\subsection{prázdné slovo - $\epsilon$}
$q_0 B \vdash\ Bq_f$

\subsection{slovo 0}
$q_0 0 \vdash 0q_1 \vdash q_2 0Z \vdash 0'q_3 Z \vdash 0'q_4 Z0 \vdash q_4 0'Z0
\vdash q_2 B0Z0 \vdash q_6 0Z0 \vdash q_6 ZO \vdash q_f 0$

\subsection{slovo 001}
$q_0 001 \vdash 0q_1 01 \vdash 00q_1 1 \vdash 001q_1 \vdash 00q_2 1Z \vdash
001'q_5 Z \vdash 001'Zq_5 B \vdash 001'Zq_4 1 \vdash 001'q_4 Z1 \vdash 00q_4
1'Z1 \vdash 0q_2 01Z1 \vdash 0'0q_3 1Z1 \vdash 0'01q_3 Z1 \vdash 0'01Zq_3 1
\vdash 0'01Z1q_3 B \vdash 0'01Zq_4 10 \vdash 0'01q_4 Z10 \vdash 0'0q_4 1Z10
\vdash 0q_4 0'1Z10 \vdash q_2 001Z10 \vdash 0'q_3 01Z10 \vdash 0'0q_3 1Z10 \vdash 0'01q_3 Z10
\vdash 0'01Zq_3 10 \vdash 0'01Z1q_3 0 \vdash  0'01Z10 q_3 B \vdash 0'01Z1q_4 00
\vdash 0'01Zq_4 100 \vdash 0'01q_4 Z100 \vdash 0'0q_4 1Z100 \vdash 0'q_4 01Z100
\vdash q_4 0'01Z100 \vdash q_2 B001Z100 \vdash q_6 001Z100 \vdash q_6 01Z100
\vdash q_6 1Z100 \vdash q_6 Z100 \vdash q_f 100 $
\end{document}
